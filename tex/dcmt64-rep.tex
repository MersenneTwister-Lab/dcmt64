\documentclass{article}
\usepackage{amssymb}
\usepackage{url}
\usepackage{amsmath}
\usepackage[dvipdfm]{graphicx}
%\usepackage{color}
%\usepackage[bottom]{footmisc}

%\newtheorem{theorem}{Theorem}[section]
%\newtheorem{conjecture}[theorem]{Conjecture}
%\newtheorem{corollary}[theorem]{Corollary}
%\newtheorem{proposition}[theorem]{Proposition}
%\newtheorem{lemma}[theorem]{Lemma}
%\newdef{definition}[theorem]{Definition}
%\newdef{remark}[theorem]{Remark}

\def\F2{{\mathbb F}_2}
%\def\wt{{\rm wt}}
%\def\wo{{\rm wt}_o}
%\def\wf{{\rm wt}_f}
%\def\UL{{\rm ul}}
%\def\bx{{{\mathbf x}}}
%\def\by{{{\mathbf y}}}
%\def\bz{{{\mathbf z}}}
%\def\bw{{{\mathbf w}}}
%\def\bu{{{\mathbf u}}}

%\def\im{{\mathrm{Im}}}
%\def\ker{{\mathrm{Ker}}}
%\def\id{{\mathrm{Id}}}
%\def\tr{{\mathrm{tr}}}

\title{Technical Report: Dynamic Creator of 64-bit Mersenne Twister}
\author{Mutsuo Saito \and Makoto Matsumoto}
\date{2019-5-13}

\begin{document}

\maketitle

\begin{abstract}
  On this report, we describe a dynamic creation program for 64-bit
  Mersenne Twister.
\end{abstract}

\keywords{
Mersenne Twister; Dynamic Creation; Pseudorandom Number; Parallel Computing
}

%\category{G.3}{Mathematics of Computing}{PROBABILITY AND STATISTICS}[Random number generation]

\section{Introduction}

Dynamic Creation of Pserudorandom Number Generators (DC)~\cite{DC} was
proposed by Matsumoto and Nishimura and a program for 32-bit Mersenne
Twister (MT)~\cite{MT} written in C is published on web page
\url{http://www.math.sci.hiroshima-u.ac.jp/~m-mat/MT/DC/dc.html} and
Git Hub \url{https://github.com/MersenneTwister-Lab/dcmt}.
64-bit Mersenne Twister (MT64)~\cite{MT64} was proposed by the same
authors of DC, Matsumoto and Nishimura, but dynamic creation program
for MT64 has not been supported.

There is a difficulty in calculating tempering parameter of MT64.  In
MT and MT64, the best tempering parameter is searched in some
condition (for tempering parameter and its calculating algorithm, see
\cite{MT, MT64}).

\begin{quote}
  In \cite{TGFSRII}\cite{MT}, our tempering strategy is back-tracking
  to obtain nealy optimal $k(v)$, but here in DC, we adopted more a
  greedy algorithm, i.e. we obtain tempering parameter from
  $v = 1, 2 ..., i,$ then $v = i + 1$ tempering parameter is searched
  so that $k(i+1)$ attains the maximal for the possible tempering
  parameters, with fixed parameter for $v = 1, ..., i$.
\end{quote}

The greeding algorithm still takes too much time in the case of 64-bit
MT.  Now, we give up the greeding algorthm and take a so-so good
altanative. we adopted ``Partial Bit Pattern'' algorithm, which is the
algorithm we used in Mersenne Twister of Graphic Processors
(MTGP)~\cite{MTGP}.  The algorithm separates tempering parameters some
bit blocks and selects a bit pattern which gives the best $k(v)$ in
the blocks. Note, we talk about searching tempering paramaeter
algorithm, the tempering algorithm itself are different from MTGP.

\section*{Speed}



\bibliographystyle{plain}
\bibliography{sfmt-kanren}
\end{document}
