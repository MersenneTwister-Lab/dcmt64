\documentclass{article}
\usepackage{amssymb}
\usepackage{amsmath}
\usepackage{url}
\usepackage{comment}
\usepackage[dvipdfm]{graphicx}
\usepackage[dvipdfm,truedimen,margin=25mm, nohead]{geometry}
%\usepackage{color}
%\usepackage[bottom]{footmisc}

%\newtheorem{theorem}{Theorem}[section]
%\newtheorem{conjecture}[theorem]{Conjecture}
%\newtheorem{corollary}[theorem]{Corollary}
%\newtheorem{proposition}[theorem]{Proposition}
%\newtheorem{lemma}[theorem]{Lemma}
%\newdef{definition}[theorem]{Definition}
%\newdef{remark}[theorem]{Remark}

\def\F2{{\mathbb F}_2}
%\def\wt{{\rm wt}}
%\def\wo{{\rm wt}_o}
%\def\wf{{\rm wt}_f}
%\def\UL{{\rm ul}}
%\def\bx{{{\mathbf x}}}
%\def\by{{{\mathbf y}}}
%\def\bz{{{\mathbf z}}}
%\def\bw{{{\mathbf w}}}
%\def\bu{{{\mathbf u}}}

%\def\im{{\mathrm{Im}}}
%\def\ker{{\mathrm{Ker}}}
%\def\id{{\mathrm{Id}}}
%\def\tr{{\mathrm{tr}}}

\title{Technical Report: Dynamic Creator of 64-bit Mersenne Twister}
\author{Mutsuo Saito \and Makoto Matsumoto}
\date{2019-8-3}

\begin{document}

\maketitle

\begin{abstract}
  On this report, we describe a dynamic creation program for 64-bit
  Mersenne Twister.
\end{abstract}

%\keywords{
%Mersenne Twister; Dynamic Creation; Pseudorandom Number; Parallel Computing
%}

%\category{G.3}{Mathematics of Computing}{PROBABILITY AND STATISTICS}[Random number generation]

\subsection*{Improvement}

Dynamic Creation of Pserudorandom Number Generators (DC)~\cite{DC} was
proposed by Matsumoto and Nishimura and a program for 32-bit Mersenne
Twister (MT)~\cite{MT} written in C is published on web page
\url{http://www.math.sci.hiroshima-u.ac.jp/~m-mat/MT/DC/dc.html} and
Git Hub \url{https://github.com/MersenneTwister-Lab/dcmt}.  64-bit
Mersenne Twister (MT64)~\cite{MT64} was proposed by Nishimura,
but dynamic creation program for MT64 has not been supported.

There was a difficulty in calculating tempering parameter of MT64.
Two tempering parameter sizes of MT are 25 and 17 bits, on the other
hand, those of MT64 are 47 and 27 bits.  Increased sizes of tempering
parameters made dynamic creation very slow and impractical. To solve
this difficulty, now, we adopt ``Partial Bit Pattern'' algorithm,
which is the algorithm used in Mersenne Twister of Graphic Processors
(MTGP)~\cite{MTGP}.  The algorithm separates tempering parameters some
bit blocks and selects a bit pattern which gives the best $k(v)$ in
the blocks. (Note: we talk about the algorithm of searching tempering
paramaeters. The tempering algorithm itself of MT64 is not changed.)

\subsection*{Results of the Dynamic Creation}

We made C++ program
dcmt64(\url{https://github.com/MersenneTwister-Lab/dcmt64}).  We
executed the program on Super Computer of The Institute of Statistical
Mathematics(ISM) in Japan.

\begin{center}
  Execution Environment:\\
  \medskip
\begin{tabular}{rl} \hline
  CPU & Intel Xeon Gold 6154 (18 core, 3.0GHz)\\
  Memory & 384GB \\
  OS & Red Hat Linux Enterprise Server 7\\
  Compiler & g++ \\
  \hline
\end{tabular}

\bigskip
Execution Results:\\
\medskip
\begin{tabular}{ll|ll} \hline
  used time & 10h & used process & 36 \\ \hline
  specified mexp & 19937 & parameters found & 1161 \\
  \hline
\end{tabular}
\end{center}

Dimension defect(DD)\cite[\S 1.2]{MT} is a simple evaluation index of
$F2$-linear random number generators. The less DD is the better.
The minimum DD of found parameters is 5023 and the maximum one
is 25258. 95 percent parameters have DD less than 7000.

\bibliographystyle{plain}
\bibliography{sfmt-kanren}
\end{document}
